%chktex 44

\documentclass[fleqn]{protokol}
\usepackage{array}
\usepackage{tabularx}
\usepackage{environ}

\usepackage[style=iso-numeric]{biblatex}
\usepackage{csquotes}
\addbibresource{ref.bib}
%------------------- Zde vyplňte údaje -------------------------------
\autor{Václav Horáček}
\autorID{256296}
\autorr{Jan Holík}
\autorrID{256295}
\rocnik{3}
\merenodne{25.\,11.\,2025}
\nazev{Měření s odporovými tenzometry}
\predmet{Snímače}
\teplota{22.2}
\tlak{969.9}
\vlhkost{32.6}
%=====================================================================

\newcommand{\neweq}{\\[0.8ex]}

\begin{document}

\maketitle                  % Vygeneruje titulní stránku podle vyplněných údajů
\tableofcontents\newpage   % Vygeneruje obsah
%------------------- Zde začíná samotný dokument ---------------------

% command pro psani promennych k rovnicim

\NewEnviron{conditions}{%
    \text{Kde je:}
    \noindent
    \begin{table}[!h]
        \begin{tabular}{@{}ll}
        \BODY
        \end{tabular}
    \end{table}
    \linebreak
}


% Uprava tabulek
\def\arraystretch{1.3}
\setlength{\headheight}{15pt}
\renewcommand{\sectionmark}[1]{\markboth{#1}{}}

\newcolumntype{C}{>{\centering\arraybackslash}X}

% FUNKCE PRO GENEROVANI TABULEK 
%   1. argument popisek
%   2. argument format
%   3. argument label
%   Do tela psat data a \hline
\NewEnviron{protocoltable}[3][Tabulka]{%
    \begin{table}[!h]
        \centering
        \caption{#1}\label{tab:#3}
        \vspace{0.3cm}
        \begin{tabularx}{\textwidth}{#2}
            \BODY
        \end{tabularx}
    \end{table}
}

% FUNKCE PRO TISK OBRAZKU
%  1. argument titulek
%  2. argument cesta napr. src\neco.png
%  3. argument scale - velikost rozsah 0.0-1.0
%  4. argument label
\newcommand{\printfigure}[4][Obrazek]{%
    \begin{figure}[!h]
        \centering
        \includegraphics[scale=#3]{#2}
        \caption{#1}
        \label{obr:#4}
    \end{figure}
}


\section{Zadání}\label{kap:zadani}
    \begin{enumerate}
        \item Seznamte se s~technickým popisem k~tenzometrické aparatuře Vishay P-3500, její obsluhou a~s~popisem přístroje pro~zjišťování citlivosti tenzometrů.     
        \item Změřte hodnotu součinitele deformační citlivosti K~tenzometru nalepeného na~přípravku pro~cejchování tenzometrů a~porovnejte ji~s~údaji výrobce. Z~naměřených hodnot vypočítejte i~přírůstek odporu tenzometru při~průhybu 1 mm. Odpor nezatíženého tenzometru je $R = 120\ \Omega$.
        \item Zjistěte rezonanční kmitočty malých vetknutých nosníčků pomocí nalepených tenzometrů.
        \item Zjistěte modul pružnosti~$E$, Poissonova číslo~$\mu$ materiálu velkého nosníku a~hmotnost závěsu závaží u~velkého nosníku.
        \item Zjistěte modul pružnosti~$E$ materiálu velkého nosníku a~hmotnost závěsu závaží při~využití podelně nalepeného tenzometru. Využijte přípravky \linebreak můstkových zapojení, zdroj a~voltmetr jako náhradu ústředny Vishay P-3500 pro varianty: čtvrtmůstek s~dvouvodičovým zapojením, čtvrtmůstek s~třívodičovým zapojením a~půlmůstek. Napájecí napětí můstků nastavte stejné jako u~aparatury Vishay P-3500. Vyhodnoťte schopnost jednotlivých zapojení potlačit vliv odporu přívodních vodičů simulovaných vloženým odporem.
        \item V konfiguraci třívodičového zapojení jako čtvrmůstek spočítejte ze~známých hodnot odporů a~napájecího napětí výstupní napětí můstku a~stanovte nejistotu tohoto výstupního napětí. Vypočítanou teoretickou hodnotu napětí porovnejte s~naměřenou hodnotou výstupního napětí čtvrtmůstku při~nezatíženém nosníku.
    \end{enumerate}

\pagebreak 

% Ukol 1 - 
\section{Úkol 1 - Popis aparatury Vishay P-3500 a její obshuha}
 
    \subsection{Zpracování}

    \begin{flushleft}
        \textbf{Aparatura Vishay P-3500:}
    \end{flushleft}
    
    \begin{flushleft}
    \textit{Vishay P-3500} slouží k měření deformací pomocí tenzometrů. Umožňuje měření s jedním tenzometrem, se dvěma tenzometry, nebo s plným mostem. Displej zobrazuje naměřené hodnoty v hodnotách relativního prodloužení, tzv. microstrainech (1$\mu$m/m = $10^{-6}$).
    \end{flushleft}

    \printfigure[Přední panel Vishay P-3500]{src/Vishay_P-3500.jpg}{0.15}{ukol1-Vishay P-3500}

    \begin{flushleft}
        \textit{POWER OFF} - modrá výplň tlačítka. Slouží k vypnutí napájení aparatury.\\[4mm]
        
        \textit{AMP ZERO} - oranžová výplň tlačítka. Používá se pro nastavení vyvážení zesilovače měřícího můstku.\\[4mm]

        \textit{GAGE FACTOR} - oranžová výplň tlačítka. Slouží k nastavení činitele deformační citlivosti v rozsahu 0,500 až 9,900.\\[4mm]

        \textit{RUN} - zelená výplň tlačítka. Měřící ústředna je ve stavu měření.\\[4mm]

        \textit{BRIDGE} - plný nebo 1/4, 1/2 můstek – žlutá výplň tlačítka. Tímto tlačítkem nastavíme zapojení tenzometrů celý můstek nebo 1/4, 1/2 -můstek.\\[4mm]
    
        \textit{OUTPUT} - analogový výstup ze zesilovače.\\[4mm]

        \textit{BATTERY} - ukazuje stav nabití vnitřní baterie.
    \end{flushleft}
    \pagebreak

    \printfigure[Možná zapojení Vishay P-3500]{src/Vishay_zapojeni.png}{0.8}{ukol1-Vishay P-3500}

    \begin{flushleft}
        \textbf{Přístroj pro zjišťování citlivosti tenzometrů:}
    \end{flushleft}
    
    \printfigure[Možná zapojení Vishay P-3500]{src/Pristroj_citlivost.png}{0.3}{ukol1-Vishay P-3500}

    \begin{flushleft}
        Používá se pro zjišťování součinitele deformační citlivosti K odporových tenzometrů s odměrnou délkou do 50mm. Platí:

        \begin{equation}
            \frac{\Delta R}{R} = K \cdot \frac{\Delta l}{l} = K \cdot \epsilon
        \end{equation}

        Cejchování se provádí na duralovém nosníku ohýbaném ve velkém rozsahu konstantním ohybovým momentem. Část nosníku ve vzdálenosti b mezi podporami má konstantní ohybový moment a tím též konstantní poměrné prodloužení povrchových vláken. 
        Při lepení tenzometrů je nutné se vyhnout přímému okolí podpor. 
    \end{flushleft}

    \pagebreak

    \begin{flushleft}
        Pro nosník, jehož deformační křivkou je kružnice o poloměru r, platí pro poměrnou deformaci povrchového vlákna o délce l:
    \end{flushleft}
    
    \begin{equation}
        r = \frac{b^2}{8y} \qquad \frac{\Delta l}{0{,}5h} = \frac{l}{r} \quad \Rightarrow \quad \epsilon = \frac{4h}{b^2}y
    \end{equation}

    \begin{flushleft}
        \textit{y} - průhyb nosníku [m]
    \end{flushleft}
    
   
    
    \subsection{Závěr}  
    V této úloze bylo provedeno seznámení s aparaturou Vishay P-3500 a přístrojem pro zjišťování citlivosti tenzometrů. Aparatura je určena k měření deformací pomocí tenzometrů a umožňuje měření s jedním tenzometrem, se dvěma tenzometry, nebo s plným mostem

\pagebreak

% Ukol 2 - 
\section{Úkol 2 - Součinitele deformační citlivosti}

    \subsection{Teoretický rozbor}
        Vztah pro relativní změnu odporu na poměrné deformaci $\epsilon$ kovových tenzometrů lze vyjádřit jako:
        \begin{equation}
            \label{hlavni rovnice}
            \dfrac{\Delta R}{R} = K \cdot \dfrac{\Delta l}{l} = K \cdot \epsilon \quad[-]
        \end{equation}
        Přičemž relativní změna odporu tenzometru je možná vyjádřit jako:
        \begin{equation}
            \label{deltaR}
            \dfrac{\Delta R}{R} = 4 \cdot \dfrac{\Delta U_2}{U_1} = 4 \cdot \dfrac{U_2 - U_{20}}{U_1} [-]
        \end{equation}
        Pro získání informace o $U_2$ lze využít vztah:
        \begin{equation}
            \label{napeti}
            U_2 = \dfrac{U_1\cdot D \cdot 10^{-6} \cdot GF \cdot MULT}{Z}\quad [V]
        \end{equation}
        \begin{conditions}
            $U_1$  & napájecí napětí tenzometrového můstku [V] \\
            $D$    & hodnota čtená na displeji [-] \\
            $GF$   & nastavená hodnota GAGE FACTOR [mV/V] \\
            $MULT$ & koeficient zohledňující změnu zesílení [-] \\
            $Z$    & vnitřní zesílení aparatury [-]
        \end{conditions}
        Po dosazení do vztahu~(\ref{napeti}) dojde ke zjednodušení: 
        \begin{equation}
            U_2 = \dfrac{2 \cdot D \cdot 10^{-6} \cdot 1 \cdot 1}{4}\quad = 5\cdot10^{-7} D \quad [V]
        \end{equation}
        Po dalším dosazení do vztahu~(\ref{deltaR}):
        \begin{equation}
            \label{lepsideltaR}
            \dfrac{\Delta R}{R} =  4 \cdot 5\cdot 10^{-7} \cdot \dfrac{D - D_{0}}{2} = 10^{-6} (D - D_{0}) \quad [-]
        \end{equation}
        Pro poměrné relativní namáhání $\epsilon$ platí:
        \begin{equation}
            \label{epsilon}
           \epsilon = \dfrac{4h}{b^2}y = \dfrac{4 \cdot 9.7 \cdot 10^{-3}}{(200 \cdot 10^{-3})^2} \cdot y = 0.97 \cdot y \quad [-]
        \end{equation} 
        Pokud dosadíme závislosti (\ref{epsilon}) a (\ref{lepsideltaR}) do rovnice (\ref{hlavni rovnice}), vznikne závislost:
        \begin{equation}
           10^{-6} (D - D_{0}) = K \cdot 0.97 \cdot y \rightarrow \Delta D = K \cdot 0.97 \cdot 10^{6} \cdot y \quad[-] 
        \end{equation}
        K lze poté získat proměřením závislosti údajem z aparatury \textit{Vishay P-3500 D} a prohnutím nosníku \textit{y} pomocí metody nejmenších čtverců.
        \begin{equation}
            K = \left( \dfrac{n \sum_{i}^{n} \Delta D_i y_i - \sum_{i}^{n} \Delta D_i \sum_{i}^{n} y_i }{n \sum_{i}^{n} y_i^2 - \left(\sum_{i}^{n} y_i \right)^2} \right)
        \end{equation}


    \subsection{Postup měření}

    \begin{enumerate}
        \item Tenzometr na kalibračním přípravku byl připojen k Vishay P-3500.
        \item Byla odečtena hodnota zobrazená na displeji aparatury při nulovém průhybu.
        \item Průhyb byl nastaven na 0,1 mm.
        \item Byla odečtena hodnota zobrazená na displeji aparatury.
        \item Toto bylo opakováno pro průhyby do 1 mm a zpět s krokem 0,1 mm.
    \end{enumerate}

    \subsection{Naměřené hodnoty}

        \begin{protocoltable}[Naměřené údaje z cejchovací aparatury]{|c|C|C|C|C|C|C|}{ukol2-mereni}
            \hline
            $l$[mm] & 0 & 0.1 & 0.2 & 0.3 & 0.4 & 0.5  \\ 
            \hline
            $D_\downarrow$[$\mu$m$\cdot$m$^{-1}$] & 284 & 78 & -130 & -340 & -551 & -764  \\  
            \hline
            $D_\uparrow$[$\mu$m$\cdot$m$^{-1}$] & 295 & 86 & -122 & -333 & -546 & -761  \\  
            \hline
            \hline
            $l$[mm] & 0.6 & 0.7 & 0.8 & 0.9 & 1 & X  \\ 
            \hline
            $D_\downarrow$[$\mu$m$\cdot$m$^{-1}$] & -974 & -1183 & -1388 & -1595 & -1800 & X \\  
            \hline
            $D_\uparrow$[$\mu$m$\cdot$m$^{-1}$] & -970 & -1180 & -1386 & -1595 & -1800 & X \\  
            \hline
        \end{protocoltable}


    \subsection{Zpracované výsledky měření}
        \begin{equation*}
            \Delta D = \left| D_{\uparrow} - D_{\uparrow 0} \right| = \left| -1800 - 284 \right| = 2084 \  \mu m / m
        \end{equation*}
        \begin{equation*}
            y = 0.97 \cdot l = 0.97 \cdot 0.001 = 0.00097 \ m = 0.970 \ mm
        \end{equation*}

        %0	0.097 0.194 0.291	0.388	0.485	0.582	0.679	0.776	0.873	0.970
        % 0 & 206 &	414 & 624 &	835 & 1048 & 1258 & 1467 & 1672 & 1879 & 2084
        % 0	& 209 &	417 & 628 &	841 & 1056 & 1265 & 1475 & 1681 & 1890 & 2095
        \begin{protocoltable}[Vypočtené hodnoty pro výpočet $K$]{|c|C|C|C|C|C|C|}{ukol2-deformace}
            \hline
            $y$[mm] & 0 & 0.097 & 0.194 & 0.291 & 0.388 & 0.485  \\ 
            \hline
            $\Delta D_\downarrow$[$\mu$m$\cdot$m$^{-1}$] &  0 & 206 &	414 & 624 &	835 & 1048   \\  
            \hline
            $\Delta D_\uparrow$[$\mu$m$\cdot$m$^{-1}$] & 0	& 209 &	417 & 628 &	841 & 1056  \\  
            \hline
            \hline
            $l$[mm] & 0.582 & 0.679 & 0.776 & 0.873 & 0.970 & X  \\ 
            \hline
            $\Delta D_\downarrow$[$\mu$m$\cdot$m$^{-1}$] & 1258 & 1467 & 1672 & 1879 & 2084 & X \\  
            \hline
            $\Delta D_\uparrow$[$\mu$m$\cdot$m$^{-1}$] & 1265 & 1475 & 1681 & 1890 & 2095 & X \\  
            \hline
        \end{protocoltable}

        \begin{align*}
            K_\downarrow &= \left( \dfrac{n \sum_{i}^{n} \Delta D_i l_i - \sum_{i}^{n} \Delta D_i \sum_{i}^{n} l_i }{n \sum_{i}^{n} l_i^2 - \left(\sum_{i}^{n} l_i \right)^2} \right) \\
            K_\downarrow &= \dfrac{11 \cdot 7.802 \cdot 10^{-6} - 5.335\cdot 10^{-3} \cdot 1.156\cdot 10^{-2}}{11 \cdot 3.622\cdot 10^{-6} \cdot 2.846e{-5}} \rightarrow \boxed{K = 2.155}
        \end{align*}

        Pro výchylku směrem nahoru vyšlo $\boxed{K_\uparrow = 2.166}$ a byla vypočítána stejně jako $K_\uparrow$. Průměr těchto dvou hodnot je tedy:

        \begin{equation*}
            \overline{K} = \dfrac{K_\downarrow + K_\uparrow}{2} = \dfrac{2.155 + 2.166}{2} \rightarrow \boxed{K = 2.160}
        \end{equation*}
    \subsection{Závěr} 
    
\pagebreak

\section{Úkol 3 - Rezonanční kmitočty malých nosníčků}

    \subsection{Teoretický rozbor}
    \subsection{Postup měření}

    \begin{enumerate}
        \item K aparatuře Vishay P-3500 byly připojeny tenzometry na malých nosnících.
        \item K osciloskopu byl na první kanál přiveden analogový výstup z aparatury Vishay P-3500 a na druhý kanál byl přiveden signál z generátoru.
        \item Na osciloskopu byl nastaveno zobrazení X-Y, tedy Lissajousových obrazců a pomocí nich byl nalezen rezonanční kmitočet nosníků.
    \end{enumerate}

    \subsection{Naměřené hodnoty}

        \begin{table}[!h]
            \centering
            \caption{Naměřené rezonanční frekvencne malých nosníčků}\label{tab:ukol3-frekvence}
            \vspace{0.3cm}
            \begin{tabular}{|c|c|}
                \hline
                $f$[Hz] & 46.95  \\ 
                \hline
                $f$[Hz] & 58.82  \\ 
                \hline
            \end{tabular}
        \end{table}

    \subsection{Zpracované výsledky měření}
    \subsection{Závěr} 

\pagebreak

% Ukol 4 - 
\section{Úkol 4 - Parametry velkého nosníku}

    \subsection{Teoretický rozbor}
        \begin{equation}
            \sigma = E \cdot \epsilon
        \end{equation}
        \begin{conditions}
            $\sigma_n$ & mechanické napětí v příslušném směru [Pa] \\
            $E$ & modul pružnosti v tahu [Pa]. Je to konstanta úměrnosti, která je u běžných  \\
                & kovových materiálů směrově nezávislá \\
            $\epsilon$ & je poměrná deformace v příslušném směru [-] 
        \end{conditions}
        \begin{equation}
            \mu = \dfrac{\epsilon_t}{\epsilon_n}
        \end{equation}
        \begin{conditions}
            $\mu$ & Poissonovo číslo [-] \\
            $\epsilon_t$ & poměrná deformace ve směru kolmém na působící sílu [-] \\
            $\epsilon_n$ & poměrná deformace ve směru působící síly [-]
        \end{conditions}
        \begin{align}
            \sigma_n &= \dfrac{M}{W} \quad [\text{Pa}]\\
            W &= \dfrac{1}{6} \cdot b \cdot h^2 \quad [\text{m}^3]\\
            M &= F \cdot r \quad [N\cdot m] 
        \end{align}
        \begin{conditions}
            $\sigma_n$ & mechanické napětí [Pa] \\
            $M$ & ohybový moment [N$\cdot$m] \\
            $W$ & modul průřezu nosníku [m$^3$] \\
            $F$ & síla působící na nosník [N]
        \end{conditions}

    \pagebreak
    \subsection{Postup měření}

    \begin{enumerate}
        \item K aparatuře Vishay P-3500 byl připojen tenzometr na velkém nosníku měřící podélnou složku poměrné deformace. 
        \item Byla zapsána hodnota zobrazená na displeji pro nezatížený nosník.
        \item Na nosník by zavěšen závěs a zapsána zobrazená hodnota na displeji.
        \item K závěsu bylo přidáno závaží 1 kg a zapsána zobrazená hodnota na displeji.
        \item Tímto zůsobem bylo postupováno s přidáváním závaží vždy po 1 kg až do stavu, kdz na nosníku byl zavěšen závěs a 5 kg.
    \end{enumerate}

    \subsection{Naměřené hodnoty}
        \begin{itemize}
            \item $r = 0.4$ m
            \item $b = 0.08$ m
            \item $h = 0.006$ m
        \end{itemize}
        %   m4 = [0 2.78 3.78 4.78 5.78 6.78 7.78];
        %   delta_l4_podel = [170 388 463 553 630 698 781];
        %   delta_l4_pric = [-156 -209 -231 -254 -277 -298 -320];

        \begin{protocoltable}[Naměřené hodnoty z Vishay P-3500 při postupném zatěžování nosníku]{|c|C|C|C|C|C|C|C|}{ukol4-mereni}
            \hline
            $m$[kg] & 0 & $m_z$ & $m_z+1$ & $m_z+2$ &  $m_z+3$ & $m_z+4$ & $m_z+5$  \\ 
            \hline
            $D_{n}$[$\mu$m$\cdot$m$^{-1}$] & 170 & 388 & 463 & 553 & 630 & 698 & 781  \\  
            \hline
            $D_{t}$[$\mu$m$\cdot$m$^{-1}$] & -156 & -209 & -231 & -254 & -277 & -298 & -320  \\  
            \hline
        \end{protocoltable}

    \subsection{Zpracované výsledky měření}

        \printfigure[Závislost údaje Vishay P-3500 $D$ na zatížení nosníku $m$]{src/zatizeni.png}{0.35}{charka}

        

        \begin{align*}
            M &= F \cdot r = m \cdot g \cdot r = 7.78 \cdot 9.81 \cdot 0.4 = 30.50 \ N \cdot \text{m} \\
            W &= \dfrac{1}{6} \cdot b \cdot h^2 = \dfrac{1}{6} \cdot 0.08 \cdot 0.006^2 = 4.8 \cdot 10^{-7} \ \text{m}^3 \\
            \sigma_n &= \dfrac{M}{W} = \dfrac{10.9}{4.8 \cdot 10^{-7}} = 6.354\cdot 10^7 \ \text{Pa}
        \end{align*}

        \begin{equation*}
            \epsilon_{n} = \dfrac{\Delta D_{n} \cdot 10^{-6}}{K} = \dfrac{|D_{n} - D_{n0}| \cdot 10^{-6}}{K} = \dfrac{|781 - 170| \cdot 10^{-6}}{2.16} = 2.829 \cdot 10^{-4} \\
        \end{equation*}

        \begin{equation*}
              \epsilon_{t} = \dfrac{\Delta D_{t} \cdot 10^{-6}}{K} = \dfrac{|D_{t} - D_{t0}| \cdot 10^{-6}}{K} = \dfrac{|-320 + 156| \cdot 10^{-6}}{2.16} = 7.591 \cdot 10^{-5} 
        \end{equation*}

        \begin{equation*}
            E = \dfrac{\sigma_n}{\epsilon_n} = \dfrac{6.354\cdot 10^7}{2.829 \cdot 10^{-4}} = 2.243 \cdot 10^{11} \  \text{Pa}
        \end{equation*}

        \begin{equation*}
            \mu = \dfrac{\epsilon_t}{\epsilon_n} = \dfrac{7.591 \cdot 10^{-5}}{2.829 \cdot 10^{-4}} = 0.2684 
        \end{equation*}
        % 0 & 2.270 \cdot 10^{7} & 3.087\cdot 10^{7} & 3.904 \cdot 10^{7} & 4.720 \cdot 10^{7} & 5.537\cdot 10^{7} & 6.354 \cdot 10^{7}
        % 0	& 1.009 & 1.356 & 1.773 & 2.129 & 2.444 & 2.828
        % 0 & 24.53 & 34.71 & 45.36 & 56.01 & 65.73 & 75.91
        % X & 225.0 & 227.6 & 220.2	& 221.6 & 226.6 & 224.7
        \begin{protocoltable}[Vypočtené hodnoty $E$, $\mu$ a veličiny pro to potřebné]{|c|C|C|C|C|C|C|C|}{ukol4-hodnoty}
            \hline
            $m$[kg] & 0 & 2.78 & 3.78 & 4.78 & 5.78 & 6.78 & 7.78  \\ 
            \hline
            $D_{n}$[$\mu$m$\cdot$m$^{-1}$] & 170 & 388 & 463 & 553 & 630 & 698 & 781  \\  
            \hline
            $D_{t}$[$\mu$m$\cdot$m$^{-1}$] & -156 & -209 & -231 & -254 & -277 & -298 & -320  \\  
            \hline
            $M$[N$\cdot$m] & 0 & 10.90 & 14.82 & 18.74 & 22.66 & 26.58 & 30.50  \\  
            \hline
            $\sigma_n$[GPa] & 0 & $0.0227$ & $0.03087$ & $0.03904$ & $0.04720$ & $0.05537$ & $0.06354$  \\  
            \hline
            $\epsilon_{n}$[$10^{-6}$] & 0	& 100.9 & 135.6 & 177.3 & 212.9 & 244.4 & 282.8  \\  
            \hline
            $\epsilon_{t}$[$10^{-6}$] & 0 & 24.53 & 34.71 & 45.36 & 56.01 & 65.73 & 75.910  \\  
            \hline
            $E$[GPa] & X & 225.0 & 227.6 & 220.2 & 221.6 & 226.6 & 224.7  \\  
            \hline
            $\mu$[-] & X & 0.2431 & 0.2560 & 0.2559 & 0.2630 & 0.2689 &	0.2684  \\  
            \hline
        \end{protocoltable}

        \begin{align*}
            \overline{E} &= \dfrac{1}{N} \sum_{i = 0}^{n} E_i = \dfrac{1}{6} \sum_{i = 1}^{6} E = 224.3 \text{GPa} \\
            \overline{\mu} &= \dfrac{1}{N} \sum_{i = 0}^{n} \mu_i = \dfrac{1}{6} \sum_{i = 1}^{6} \mu_i = 0.2592 
        \end{align*}



    \subsection{Závěr} 
       
\pagebreak

\section{Úkol 5 - Parametry velkého nosníku, můstková měření}

    \subsection{Teoretický rozbor}
    \subsection{Postup měření}

    \begin{enumerate}
        \item Napájecí napětí můstku byla nastaveno na 2 V.
        \item Pro dvouvodičové zapojení čtvrtmůstku byly změřeny hodnoty napětí na výstupu můstků při postupném zatěžování nosníku viz. úkol 4.
        \item To samé jsme opakovali pro třívodičové zapojení čtvrtmůstku a pro půlmůstek.
    \end{enumerate}

    \subsection{Naměřené hodnoty}
        \begin{protocoltable}[Naměřená napětí z můstků při zatěžování nosníku]{|c|C|C|C|C|C|C|C|}{ukol5-napeti}
            \hline
            $m$[kg] & 0 & 2.78 & 3.78 & 4.78 & 5.78 & 6.78 & 7.78  \\ 
            \hline
            $U_{1/4} (2v)$[mV] & 0.954 & 0.851 & 0.814 & 0.774 & 0.733 & 0.683 & 0.642  \\  
            \hline
            $U_{1/4} (2v)$ s $R$ [mV] & -39.91 & -39.99 & -40.01 & -40.04 & -40.06 & -40.09 & -40.12  \\  
            \hline
            $U_{1/4} (3v)$[mV] & 1.362 & 1.260 & 1.224 & 1.192 & 1.139 & 1.105 & 1.075  \\  
            \hline
            $U_{1/4} (3v)$ s $R$ [mV] & 1.267 & 1.164 & 1.128 & 1.090 & 1.053 & 1.018 & 0.978 \\  
            \hline
            $U_{1/2}$[mV] & -0.119 & -0.251 & -0.297 & -0.346 & -0.396 & -0.442 & -0.492  \\  
            \hline
            $U_{1/2}$ s $R$ [mV] & -0.059 & -0.192 & -0.236 & -0.284 & -0.331 & -0.376 & -0.424  \\  
            \hline
        \end{protocoltable}
    \subsection{Zpracované výsledky měření}

        \begin{table}[!h]
            \centering
            \caption{Odečtené hodnoty hmotností závěsu pro různé můstky}\label{tab:ukol5-hmotnosti}
            \vspace{0.3cm}
            \begin{tabular}{|c|c|}
                \hline
                Typ můstku & Odečtená hmotnost $m$ [kg]  \\ 
                \hline
                Dvouvodičové zapojení čtvrtmůstku & 2.35  \\ 
                \hline
                -//- s předřadným R & 2.94  \\ 
                \hline
                Třívodičové zapojení čtvrtmůstku & 2.64  \\ 
                \hline
                -//-  s předřadným R & 2.76  \\ 
                \hline
                Půlmůstek & 2.72  \\ 
                \hline
                Půlmůstek s předřadným R & 2.84  \\ 
                \hline
            \end{tabular}
        \end{table}

        \begin{equation*}
            \overline{m} = \dfrac{1}{N} \sum_{i = 0}^{n} m_i = \dfrac{1}{6} \sum_{i = 0}^{6} m_i = 2.708 \text{ kg}
        \end{equation*}

        \begin{equation*}
            4 \cdot \dfrac{|U_2 - U_{20}|}{U_1} = K \cdot \epsilon \rightarrow \epsilon = 4 \cdot \dfrac{|U_2 - U_{20}|}{K \cdot U_1}
        \end{equation*}

        % 0 & 95.35 & 129.6 & 166.6 & 204.6 & 250.9 & 288.8
        % 0	& 74.06 & 92.57 & 120.3 & 138.9 & 166.6 & 194.4
        % 0 & 94.42 & 127.8 & 157.4 & 206.4 & 237.9 & 265.7
        % 0 & 95.35 & 128.7 & 163.9 & 198.1 & 230.5 & 267.5
        % 0 & 122.2 & 164.8 & 210.1 & 256.4 & 299.0 & 345.3
        % 0 & 123.1 & 163.9 & 208.3 & 251.8 & 293.5 & 337.9
        
        \begin{protocoltable}[Vypočtené hodnoty $\epsilon$ z hodnot z tabulky~\ref{tab:ukol5-napeti}]{|c|C|C|C|C|C|C|C|}{ukol5-deformace}
            \hline
            $\sigma_n$[GPa] & 0 & $0.0227$ & $0.03087$ & $0.03904$ & $0.04720$ & $0.05537$ & $0.06354$  \\   
            \hline
            $\epsilon_{1/4} (2v)$[$10^{-6}$] & 0 & 95.35 & 129.6 & 166.6 & 204.6 & 250.9 & 288.8  \\  
            \hline
            $\epsilon_{1/4} (2v)$ s $R$ [$10^{-6}$] & 0	& 74.06 & 92.57 & 120.3 & 138.9 & 166.6 & 194.4  \\  
            \hline
            $\epsilon_{1/4} (3v)$[$10^{-6}$] & 0 & 94.42 & 127.8 & 157.4 & 206.4 & 237.9 & 265.7  \\  
            \hline
            $\epsilon_{1/4} (3v)$ s $R$ [$10^{-6}$] & 0 & 95.35 & 128.7 & 163.9 & 198.1 & 230.5 & 267.5 \\  
            \hline
            $\epsilon_{1/2}$[$10^{-6}$] & 0 & 122.2 & 164.8 & 210.1 & 256.4 & 299.0 & 345.3  \\  
            \hline
            $\epsilon_{1/2}$ s $R$ [$10^{-6}$] & 0 & 123.1 & 163.9 & 208.3 & 251.8 & 293.5 & 337.9  \\  
            \hline
        \end{protocoltable}

        % X & 238.1 & 238.2 & 234.3 & 230.7 & 220.7 & 220.0
        % X & 306.6 & 333.5 & 324.4 & 340.0 & 332.3 & 326.8
        % X & 240.4 & 241.6 & 248.0 & 228.7 & 232.7 & 239.1
        % X & 238.1 & 239.9 & 238.2 & 238.3 & 240.2 & 237.5
        % X & 234.0 & 236.0 & 233.9 & 231.8 & 233.2 & 231.7
        % X & 232.2 & 237.2 & 236.0 & 236.1 & 237.6 & 236.8

        \begin{equation*}
            E_{1/4}= \dfrac{\sigma_n}{\epsilon}
        \end{equation*}

        \begin{equation*}
            E_{1/2} = (1+\mu) \cdot \dfrac{\sigma_n}{\epsilon}
        \end{equation*}

        \begin{protocoltable}[Vypočtené hodnoty $E$ z hodnot z tabulky~\ref{tab:ukol5-deformace}]{|c|C|C|C|C|C|C|C|}{ukol5-moduly}
            \hline
            $E_{1/4} (2v)$[$GPa$] & X & 238.1 & 238.2 & 234.3 & 230.7 & 220.7 & 220.0  \\  
            \hline
            $E_{1/4} (2v)$ s $R$ [$GPa$] & X & 306.6 & 333.5 & 324.4 & 340.0 & 332.3 & 326.8 \\  
            \hline
            $E_{1/4} (3v)$[$GPa$] & X & 240.4 & 241.6 & 248.0 & 228.7 & 232.7 & 239.1  \\  
            \hline
            $E_{1/4} (3v)$ s $R$ [$GPa$] & X & 238.1 & 239.9 & 238.2 & 238.3 & 240.2 & 237.5 \\  
            \hline
            $E_{1/2}$[$GPa$] & X & 234.0 & 236.0 & 233.9 & 231.8 & 233.2 & 231.7 \\  
            \hline
            $E_{1/2}$ s $R$ [$GPa$] & X & 232.2 & 237.2 & 236.0 & 236.1 & 237.6 & 236.8  \\  
            \hline
        \end{protocoltable}

        
                % 230.3 327.2 238.4 238.7 233.4 236.0
        \begin{table}[!h]
            \centering
            \caption{foo}\label{tab:ukol5-modely-pruznosti}
            \vspace{0.3cm}
            \begin{tabular}{|c|c|}
                \hline
                Typ můstku & Spočtený model pružnosti $E$ [GPa]  \\ 
                \hline
                Dvouvodičové zapojení čtvrtmůstku & 230.3  \\ 
                \hline
                -//- s předřadným R & 327.2  \\ 
                \hline
                Třívodičové zapojení čtvrtmůstku & 238.4  \\ 
                \hline
                -//-  s předřadným R & 238.7  \\ 
                \hline
                Půlmůstek & 233.4  \\ 
                \hline
                Půlmůstek s předřadným R & 236.0  \\ 
                \hline
            \end{tabular}
        \end{table}

        %

    \subsection{Závěr} 
       

\pagebreak

\section{Úkol 6 - Nejistota výstupního napětí}
 
    \subsection{Teoretický rozbor}
        \begin{equation}
            U = f(U_z, R_1, R_2, R_3, R_t) = U_z \cdot \left( \dfrac{R_1}{R_1+R_2} - \dfrac{R_t}{R_3+R_t} \right) \quad [\text{mV}]
        \end{equation}
    \subsection{Postup měření}

    \begin{enumerate}
        \item Bylo proměřeno napájecí napětí můstku pomocí multimetru.
        \item V konfiguraci třívodičového zapojení čtvrtmůstku bylo změřeno výstupní napětí můstku pomocí multimetru.
        \item Byly změřeny jednotlivé odpory v můstku pomocí multimetru.
        \item Byl změřen odpor nezatíženého tenzometru pomocí multimetru.
    \end{enumerate}

    \subsection{Naměřené hodnoty}
            \begin{table}[!h]
            \centering
            \caption{Naměřené hodnoty výstupního napětí, napájecího napětí a odporů v můstku}\label{tab:nejistota6}
            \vspace{0.3cm}
            \begin{tabular}{|c|c|}
                \hline
                $U$[mV] & 1.445  \\ 
                \hline
                $U_z$[V] & 1.999  \\ 
                \hline
                $R_t$[$\Omega$] & 119.84  \\ 
                \hline
                $R_1$[$\Omega$] & 120.13  \\ 
                \hline
                $R_2$[$\Omega$] & 120.14  \\ 
                \hline
                $R_3$[$\Omega$] & 120.18  \\ 
                \hline
            \end{tabular}
        \end{table}
    \subsection{Zpracované výsledky měření}

    \begin{align*}
        U_{teor} &= U_z \cdot \left( \dfrac{R_1}{R_1+R_2} - \dfrac{R_t}{R_3+R_t} \right)\\
        U_{teor} &= 1.999 \cdot \left( \dfrac{120.13}{120.13+120.14} - \dfrac{119.84}{120.18+199.84} \right) \rightarrow \boxed{U_{teor} = 1.374 \cdot 10^{-3} \ \text{V}}
    \end{align*}

    \begin{align*}
        u_B(U_z) &= \dfrac{\Delta U_z}{\chi} = \dfrac{\dfrac{\delta_{UH}}{100} \cdot U_z +\dfrac{\delta_{UR}}{100} \cdot U_R}{\chi} \\
        u_B(U_z) &= \dfrac{\dfrac{0.0035}{100} \cdot 1.999 +\dfrac{0.0005}{100} \cdot 10}{\sqrt{3}} = 6.926 \cdot 10^{-5} \ \text{V}
    \end{align*}

    \begin{align*}
        u_B(R_1) &= \dfrac{\Delta R_1}{\chi} = \dfrac{\dfrac{\delta_{RH}}{100} \cdot R +\dfrac{\delta_{RR}}{100} \cdot U_R}{\chi} \\
        u_B(R_1) &= \dfrac{\dfrac{0.010}{100} \cdot 120.13 +\dfrac{0.001}{100} \cdot 1000}{\sqrt{3}} = 1.270 \cdot 10^{-2} \ \text{V}
    \end{align*}

    \begin{align*}
        A_{U_z} &=  \dfrac{\partial f}{\partial U_z} = \left( \dfrac{R_1}{R_1+R_2} - \dfrac{R_t}{R_3+R_t} \right) = 6.875 \cdot 10^{-4}\\
        A_{R_1} &=  \dfrac{\partial f}{\partial R_1} = U_z \cdot \dfrac{R_2}{{(R_1 + R_2)^2}} = 4.160\cdot 10^{-3} \ \text{V} \cdot \Omega^{-1} \\
        A_{R_2} &=  \dfrac{\partial f}{\partial R_2} = -U_z \cdot \dfrac{R_1}{{(R_1 + R_2)^2}} = -4.160\cdot 10^{-3} \ \text{V} \cdot \Omega^{-1} \\
        A_{R_3} &=  \dfrac{\partial f}{\partial R_3} = -U_z \cdot \dfrac{R_t}{{(R_3 + R_t)^2}} = -4.158\cdot 10^{-3} \ \text{V} \cdot \Omega^{-1} \\
        A_{R_t} &=  \dfrac{\partial f}{\partial R_4} = U_z \cdot \dfrac{R_3}{{(R_3 + R_t)^2}} = 4.170\cdot 10^{-3} \ \text{V} \cdot \Omega^{-1}\\
    \end{align*}

    \begin{equation*}
        u_c(U_{teor}) = \sqrt{   \sum_{i = 0}^{n} (A_i \cdot u_c(i))^2    } = 9.774 \cdot 10^{-5} \ \text{V}
    \end{equation*}

    \begin{equation*}
        U(U_{teor}) = k_r \cdot u_c(U_{teor}) = 2 \cdot 9.774 \cdot 10^{-5} = 1.955 \cdot 10^{-4} \ \text{V}
    \end{equation*}

    \begin{equation*}
        U_{teor} = (1.374 \pm 0.1955) \ \text{mV} \rightarrow \boxed{U_{teor} = (1.4 \pm 0.2) \ \text{mV}}
    \end{equation*}

    \begin{protocoltable}[Uncertainty budget pro určení výstupního napětí můstku]{|C|C|c|C|}{ukol6-budget-mustek}
            \hline
            Veličina & Hodnota & $\Delta_{MAX}$ & $\chi$ \\
            \hline 
            $U_z$[V] & $1.999$ V & $1.200 \cdot 10^{-4} \  V $ & $\sqrt{3}$\\
            \hline
            $R_1$[$\Omega$] & $120.13 \ \Omega$  &  $2.201 \cdot 10^{-2} \  \Omega $ & $\sqrt{3}$\\
            \hline
            $R_2$[$\Omega$] & $120.14 \ \Omega$  &  $2.201 \cdot 10^{-2} \  \Omega  $ & $\sqrt{3}$\\
            \hline
            $R_3$[$\Omega$] & $120.18 \ \Omega$  &  $2.201 \cdot 10^{-2} \  \Omega  $ & $\sqrt{3}$\\
            \hline
            $R_t$[$\Omega$] & $119.84 \ \Omega$  & $1.420 \cdot 10^{-2} \  \Omega  $ & $\sqrt{3}$\\
            \hline
            Veličina & $u_{B}$ & $\frac{\partial \rho}{\partial x_i}$ & $u_c$ \\
            \hline
            $U_z$[V] & $9.774 \cdot 10^{-5}$ V & $6.875 \cdot 10^{-4}$  &  $9.774 \cdot 10^{-5}$ V \\
            \hline
            $R_1$[$\Omega$] & $1.271 \cdot 10^{-2} \ \Omega$ & $4.160 \cdot 10^{-3} \  \Omega \cdot V^{-1} $ & $1.271 \cdot 10^{-2} \ \Omega$ \\
            \hline
            $R_2$[$\Omega$] & $1.271 \cdot 10^{-2} \ \Omega$ & $-4.160 \cdot 10^{-3} \  \Omega \cdot V^{-1} $ & $1.271 \cdot 10^{-2} \ \Omega$ \\
            \hline
            $R_3$[$\Omega$] & $1.271 \cdot 10^{-2} \ \Omega$ & $-4.158 \cdot 10^{-3} \  \Omega \cdot V^{-1} $ & $1.271 \cdot 10^{-2} \ \Omega$ \\
            \hline
            $R_t$[$\Omega$] & $8.195 \cdot 10^{-3} \ \Omega$ & $4.170 \cdot 10^{-3} \  \Omega \cdot V^{-1} $ & $8.195 \cdot 10^{-3} \ \Omega$ \\
            \hline
            \multicolumn{2}{|c|}{$U_{teor} = 1.445\cdot 10^{-3}$ V} & \multicolumn{2}{c|}{$u_C = 9.774 \cdot 10^{-5}$ V} \\
            \hline
        \end{protocoltable}

    \subsection{Závěr} 
    
\section{Závěr}

V prvním úkolu jsme byly seznámeni tenzometrickou aparaturou Vishay P-3500 a s popisem přístroje pro zjišťování citlivosti tenzometrů. 


\pagebreak

% Vishay P-3500, SN. 106013
% P. pro cejchování tenzometrů, SN. DKP1167
% Tenzometry, SN. DKP479
% UNI-T UTD2025C, SN. 2100000787
% Agilent 34401A, SN. MY44002660
% RC generátor Hung Chang 8204A, SN. D7064
% Instek GPD-3303S, SN. GES836987



% Seznam přístrojů
\section{Seznam použitých přístrojů}
    \begin{protocoltable}[Seznam použitých přístrojů]{|C|C|C|}{pristroje}
        \hline
        Typ & Přístroj & Inventární číslo  \\
        \hline
        Tenzometrická aparatura & Vishay P-3500 & SN. 106013 \\
        \hline
        P. pro cejchování tenzometrů &  - & SN. DKP1167 \\
        \hline
        Tenzometry & - & SN. DKP479 \\
        \hline
        Meteostanice & Testo 622 & SN. 39507568/505 \\
        \hline
        Osciloskop & UNI-T UTD2025C & SN. 2100000787 \\
        \hline
        Multimetr & Agilent 34401A & SN. MY44002660\\
        \hline
        RC generátor & Hung Chang 8204A & SN. D7064\\
        \hline
        Lab. zdroj & Instek GPD-3303S & SN. GES836987\\
        \hline


    \end{protocoltable}%

\pagebreak

% Reference
{\printbibliography}

\end{document} % Konec dokumentu
