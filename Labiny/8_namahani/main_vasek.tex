%chktex-file 44

\documentclass[fleqn]{protokol}
\usepackage{array}
\usepackage{tabularx}
\usepackage{environ}

\usepackage[style=iso-numeric]{biblatex}
\usepackage{csquotes}
\addbibresource{ref.bib}
%------------------- Zde vyplňte údaje -------------------------------
\autor{Václav Horáček}
\autorID{256296}
\autorr{Jan Holík}
\autorrID{256295}
\rocnik{3}
\merenodne{25.\,11.\,2025}
\nazev{Měření s odporovými tenzometry}
\predmet{Snímače}
\teplota{22.2}
\tlak{969.9}
\vlhkost{32.6}
%=====================================================================

\newcommand{\neweq}{\\[0.8ex]}

\begin{document}

\maketitle                  % Vygeneruje titulní stránku podle vyplněných údajů
\tableofcontents\newpage   % Vygeneruje obsah
%------------------- Zde začíná samotný dokument ---------------------

% command pro psani promennych k rovnicim

\NewEnviron{conditions}{%
    \text{Kde je:}
    \noindent
    \begin{table}[!h]
        \begin{tabular}{@{}ll}
        \BODY
        \end{tabular}
    \end{table}
    \linebreak
}


% Uprava tabulek
\def\arraystretch{1.3}
\setlength{\headheight}{15pt}
\renewcommand{\sectionmark}[1]{\markboth{#1}{}}

\newcolumntype{C}{>{\centering\arraybackslash}X}

% FUNKCE PRO GENEROVANI TABULEK 
%   1. argument popisek
%   2. argument format
%   3. argument label
%   Do tela psat data a \hline
\NewEnviron{protocoltable}[3][Tabulka]{%
    \begin{table}[!h]
        \centering
        \caption{#1}\label{tab:#3}
        \vspace{0.3cm}
        \begin{tabularx}{\textwidth}{#2}
            \BODY
        \end{tabularx}
    \end{table}
}

% FUNKCE PRO TISK OBRAZKU
%  1. argument titulek
%  2. argument cesta napr. src\neco.png
%  3. argument scale - velikost rozsah 0.0-1.0
%  4. argument label
\newcommand{\printfigure}[4][Obrazek]{%
    \begin{figure}[!h]
        \centering
        \includegraphics[scale=#3]{#2}
        \caption{#1}
        \label{obr:#4}
    \end{figure}
}


\section{Zadání}\label{kap:zadani}
    \begin{enumerate}
        \item Seznamte se s~technickým popisem k~tenzometrické aparatuře Vishay P-3500, její obsluhou a~s~popisem přístroje pro~zjišťování citlivosti tenzometrů.     
        \item Změřte hodnotu součinitele deformační citlivosti K~tenzometru nalepeného na~přípravku pro~cejchování tenzometrů a~porovnejte ji~s~údaji výrobce. Z~naměřených hodnot vypočítejte i~přírůstek odporu tenzometru při~průhybu 1 mm. Odpor nezatíženého tenzometru je $R = 120\ \Omega$.
        \item Zjistěte rezonanční kmitočty malých vetknutých nosníčků pomocí nalepených tenzometrů.
        \item Zjistěte modul pružnosti~$E$, Poissonova číslo~$\mu$ materiálu velkého nosníku a~hmotnost závěsu závaží u~velkého nosníku.
        \item Zjistěte modul pružnosti~$E$ materiálu velkého nosníku a~hmotnost závěsu závaží při~využití podelně nalepeného tenzometru. Využijte přípravky \linebreak můstkových zapojení, zdroj a~voltmetr jako náhradu ústředny Vishay P-3500 pro varianty: čtvrtmůstek s~dvouvodičovým zapojením, čtvrtmůstek s~třívodičovým zapojením a~půlmůstek. Napájecí napětí můstků nastavte stejné jako u~aparatury Vishay P-3500. Vyhodnoťte schopnost jednotlivých zapojení potlačit vliv odporu přívodních vodičů simulovaných vloženým odporem.
        \item V konfiguraci třívodičového zapojení jako čtvrmůstek spočítejte ze~známých hodnot odporů a~napájecího napětí výstupní napětí můstku a~stanovte nejistotu tohoto výstupního napětí. Vypočítanou teoretickou hodnotu napětí porovnejte s~naměřenou hodnotou výstupního napětí čtvrtmůstku při~nezatíženém nosníku.
    \end{enumerate}

\pagebreak 

% Ukol 1 - 
\section{Úkol 1 - Popis aparatury Vishay P-3500 a její obshuha}
 
    \subsection{Zpracování}

    
   
    \textit{Vishay P-3500} slouží k měření deformací pomocí tenzometrů. Umožňuje měření s jedním tenzometrem, se dvěma tenzometry, nebo s plným mostem. Displej zobrazuje naměřené hodnoty v hodnotách relativního prodloužení, tzv. microstrainech (1$\mu$m/m = $10^{-6}$).

    \printfigure[Přední panel Vishay P-3500]{src/Vishay_P-3500.jpg}{0.15}{ukol1-Vishay P-3500}

    \begin{flushleft}
        \textit{POWER OFF} - modrá výplň tlačítka. Slouží k vypnutí napájení aparatury.\\[4mm]
        
        \textit{AMP ZERO} - oranžová výplň tlačítka. Používá se pro nastavení vyvážení zesilovače měřícího můstku.\\[4mm]

        \textit{GAGE FACTOR} - oranžová výplň tlačítka. Slouží k nastavení činitele deformační citlivosti v rozsahu 0,500 až 9,900.\\[4mm]

        \textit{RUN} - zelená výplň tlačítka. Měřící ústředna je ve stavu měření.\\[4mm]

        \textit{BRIDGE} - plný nebo 1/4, 1/2 můstek – žlutá výplň tlačítka. Tímto tlačítkem nastavíme zapojení tenzometrů celý můstek nebo 1/4, 1/2 -můstek.\\[4mm]
    
        \textit{OUTPUT} - analogový výstup ze zesilovače.\\[4mm]

        \textit{BATTERY} - ukazuje stav nabití vnitřní baterie.
    \end{flushleft}
    \pagebreak

    \printfigure[Možná zapojení Vishay P-3500]{src/Vishay_zapojeni.png}{0.8}{ukol1-Vishay P-3500}

    \begin{flushleft}
        \textbf{Přístroj pro zjišťování citlivosti tenzometrů:}
    \end{flushleft}
    
    \printfigure[Přístroj pro cejchování tenzometrů~\cite{navod}]{src/Pristroj_citlivost.png}{0.3}{ejch}


        Používá se pro zjišťování součinitele deformační citlivosti K odporových tenzometrů s odměrnou délkou do 50mm. Platí:
        \begin{equation}
            \frac{\Delta R}{R} = K \cdot \frac{\Delta l}{l} = K \cdot \epsilon
        \end{equation}

    Cejchování se provádí na duralovém nosníku ohýbaném ve velkém rozsahu konstantním ohybovým momentem. Část nosníku ve vzdálenosti b mezi podporami má konstantní ohybový moment a tím též konstantní poměrné prodloužení povrchových vláken. 
    Při lepení tenzometrů je nutné se vyhnout přímému okolí podpor. 

    \pagebreak

    \begin{flushleft}
        Pro nosník, jehož deformační křivkou je kružnice o poloměru r, platí pro poměrnou deformaci povrchového vlákna o délce l:
    \end{flushleft}
    \begin{equation}
        r = \frac{b^2}{8y} \qquad \frac{\Delta l}{0{,}5h} = \frac{l}{r} \quad \Rightarrow \quad \epsilon = \frac{4h}{b^2}y
    \end{equation}
    \linebreak
    \begin{conditions}
        $y$  & průhyb nosníku [m] 
    \end{conditions}
    
   
    
    \subsection{Závěr}  
    V této úloze bylo provedeno seznámení s aparaturou Vishay P-3500 a přístrojem pro zjišťování citlivosti tenzometrů. Aparatura je určena k měření deformací pomocí tenzometrů a umožňuje měření s jedním tenzometrem, se dvěma tenzometry, nebo s plným mostem.

\pagebreak

% Ukol 2 - 
\section{Úkol 2 - Součinitele deformační citlivosti}
    Kapitola vychází ze zdroje~\cite{navod}.
    \subsection{Teoretický rozbor}
        Tenzometry se používají pro měření mechanické poměrné deformace, což je změna rozměrů objektu vůči jeho rozměrům v klidovém stavu. Tohoto údaje lze využít při měření síly, tlaku, vibrací a dalších fyzikálních veličin.
        Poměrná deformace se značí $\epsilon$ a je bezrozměrná, respektive [m/m].

        Při změně mechnických vlastností materiálu se mění i elektrické vlastnosti, zejména elektrický odpor $R$ [$\Omega$]. Tomu odpovídá rovnice:
        \begin{equation}
            \label{odpor}
            R = \rho \cdot \dfrac{l}{s} \quad [\Omega]
        \end{equation}
        \begin{conditions}
            $\rho$  & měrný elektický odpor [$\Omega \cdot$m] \\
            $l$    & délka tělesa [m] \\
            $S$   & průřez tělesa [m$^2$] 
        \end{conditions}
        Pokud budeme uvažovat v rovnici změny délky a odporu tělesa vůči jejich původním hodnotám, tak lze přejít na rovnici:
        \begin{equation}
            \label{hlavni rovnice}
            \dfrac{\Delta R}{R} = K \cdot \dfrac{\Delta l}{l} = K \cdot \epsilon \quad[-]
        \end{equation}
        Kde $K$ je \textit{součinitel deformační citlivosti} a slouží jako konstanta úměry mezi změnou elektrického odporu a mechanické poměrné deformace.
        Relativní změna odporu tenzometru je možná vyjádřit jako:
        \begin{equation}
            \label{deltaR}
            \dfrac{\Delta R}{R} = 4 \cdot \dfrac{\Delta U_2}{U_1} = 4 \cdot \dfrac{U_2 - U_{20}}{U_1} \quad [-]
        \end{equation}
        Jelikož je k měření použita aparatura \textit{Vishay P-3500}, tak je pro získání informace o $U_2$ lze využít vztah:
        \begin{equation}
            \label{napeti}
            U_2 = \dfrac{U_1\cdot D \cdot 10^{-6} \cdot GF \cdot MULT}{Z}\quad [V]
        \end{equation}
        \pagebreak
        \linebreak
        \begin{conditions}
            $U_1$  & napájecí napětí tenzometrového můstku [V] \\
            $D$    & hodnota čtená na displeji [-] \\
            $GF$   & nastavená hodnota GAGE FACTOR [mV/V] \\
            $MULT$ & koeficient zohledňující změnu zesílení [-] \\
            $Z$    & vnitřní zesílení aparatury [-]
        \end{conditions}
        Po dosazení do vztahu~(\ref{napeti}) dojde ke zjednodušení: 
        \begin{equation}
            U_2 = \dfrac{2 \cdot D \cdot 10^{-6} \cdot 1 \cdot 1}{4}\quad = 5\cdot10^{-7} D \quad [V]
        \end{equation}
        Po dalším dosazení do vztahu~(\ref{deltaR}):
        \begin{equation}
            \label{lepsideltaR}
            \dfrac{\Delta R}{R} =  4 \cdot 5\cdot 10^{-7} \cdot \dfrac{D - D_{0}}{2} = 10^{-6} (D - D_{0}) \quad [-]
        \end{equation}
        Při měření této závislosti se používá aparatura pro cejchování tenzometrů. Měření probíhá postupnou deformací nosníku při známé informaci o prohnutí. Pro poměrné relativní namáhání $\epsilon$ platí a plyne z geometrie cejchovacího přístroje:
        \begin{equation}
            \label{epsilon}
           \epsilon = \dfrac{4h}{b^2}y = \dfrac{4 \cdot 9.7 \cdot 10^{-3}}{(200 \cdot 10^{-3})^2} \cdot y = 0.97 \cdot y \quad [-]
        \end{equation} 
        Pokud dosadíme závislosti (\ref{epsilon}) a (\ref{lepsideltaR}) do rovnice (\ref{hlavni rovnice}), vznikne závislost:
        \begin{equation}
           10^{-6} (D - D_{0}) = K \cdot 0.97 \cdot y \rightarrow \Delta D = K \cdot 0.97 \cdot 10^{6} \cdot y \quad[-] 
        \end{equation}
        K lze poté získat po proměření závislosti údaje z aparatury \textit{Vishay P-3500 D} na prohnutí nosníku \textit{y} pomocí metody nejmenších čtverců~\cite{vypocty}.
        \begin{equation}
            K = \left( \dfrac{n \sum_{i}^{n} \Delta D_i y_i - \sum_{i}^{n} \Delta D_i \sum_{i}^{n} y_i }{n \sum_{i}^{n} y_i^2 - \left(\sum_{i}^{n} y_i \right)^2} \right)
        \end{equation}

       
            \pagebreak
    \subsection{Postup měření}
    \begin{enumerate}
        \item Tenzometr na kalibračním přípravku byl připojen k Vishay P-3500.
        \item Byla odečtena hodnota zobrazená na displeji aparatury při nulovém průhybu.
        \item Průhyb byl nastaven na 0,1 mm.
        \item Byla odečtena hodnota zobrazená na displeji aparatury.
        \item Toto bylo opakováno pro průhyby do 1 mm a zpět s krokem 0,1 mm.
    \end{enumerate}

    \subsection{Naměřené hodnoty}

        \begin{protocoltable}[Naměřené údaje z cejchovací aparatury]{|c|C|C|C|C|C|C|}{ukol2-mereni}
            \hline
            $l$[mm] & 0 & 0.1 & 0.2 & 0.3 & 0.4 & 0.5  \\ 
            \hline
            $D_\downarrow$[$\mu$m$\cdot$m$^{-1}$] & 284 & 78 & -130 & -340 & -551 & -764  \\  
            \hline
            $D_\uparrow$[$\mu$m$\cdot$m$^{-1}$] & 295 & 86 & -122 & -333 & -546 & -761  \\  
            \hline
            \hline
            $l$[mm] & 0.6 & 0.7 & 0.8 & 0.9 & 1 & X  \\ 
            \hline
            $D_\downarrow$[$\mu$m$\cdot$m$^{-1}$] & -974 & -1183 & -1388 & -1595 & -1800 & X \\  
            \hline
            $D_\uparrow$[$\mu$m$\cdot$m$^{-1}$] & -970 & -1180 & -1386 & -1595 & -1800 & X \\  
            \hline
        \end{protocoltable}


    \subsection{Zpracované výsledky měření}
    Pro vykreslení závislosti změny údaje aparatury $\Delta D$ na prohnutí nosníku $y$ je potřeba z naměřených hodnot $D$ změnu spočítat pomocí vztahu:
        \begin{equation*}
            \Delta D = \left| D_{\uparrow} - D_{\uparrow 0} \right| = \left| -1800 - 284 \right| = 2084 \  \mu m / m
        \end{equation*}
    Podobné je potřeba udělat i pro naměrěný údaj z cejchovacího přístroje.
        \begin{equation*}
            y = 0.97 \cdot l = 0.97 \cdot 0.001 = 0.00097 \ m = 0.970 \ mm
        \end{equation*}

        %0	0.097 0.194 0.291	0.388	0.485	0.582	0.679	0.776	0.873	0.970
        % 0 & 206 &	414 & 624 &	835 & 1048 & 1258 & 1467 & 1672 & 1879 & 2084
        % 0	& 209 &	417 & 628 &	841 & 1056 & 1265 & 1475 & 1681 & 1890 & 2095
        \begin{protocoltable}[Vypočtené hodnoty pro výpočet $K$]{|c|C|C|C|C|C|C|}{ukol2-deformace}
            \hline
            $y$[mm] & 0 & 0.097 & 0.194 & 0.291 & 0.388 & 0.485  \\ 
            \hline
            $\Delta D_\downarrow$[$\mu$m$\cdot$m$^{-1}$] &  0 & 206 &	414 & 624 &	835 & 1048   \\  
            \hline
            $\Delta D_\uparrow$[$\mu$m$\cdot$m$^{-1}$] & 0	& 209 &	417 & 628 &	841 & 1056  \\  
            \hline
            \hline
            $l$[mm] & 0.582 & 0.679 & 0.776 & 0.873 & 0.970 & X  \\ 
            \hline
            $\Delta D_\downarrow$[$\mu$m$\cdot$m$^{-1}$] & 1258 & 1467 & 1672 & 1879 & 2084 & X \\  
            \hline
            $\Delta D_\uparrow$[$\mu$m$\cdot$m$^{-1}$] & 1265 & 1475 & 1681 & 1890 & 2095 & X \\  
            \hline
        \end{protocoltable}

        Pomocí metody nejmenších čtverců je nyní možné určit součinitel deformační citlivosti $K$:
        \begin{align*}
            K_\downarrow &= \left( \dfrac{n \sum_{i}^{n} \Delta D_i l_i - \sum_{i}^{n} \Delta D_i \sum_{i}^{n} l_i }{n \sum_{i}^{n} l_i^2 - \left(\sum_{i}^{n} l_i \right)^2} \right) \\
            K_\downarrow &= \dfrac{11 \cdot 7.802 \cdot 10^{-6} - 5.335\cdot 10^{-3} \cdot 1.156\cdot 10^{-2}}{11 \cdot 3.622\cdot 10^{-6} \cdot 2.846e{-5}} \rightarrow \boxed{K = 2.155}
        \end{align*}

        Pro výchylku směrem nahoru vyšlo $\boxed{K_\uparrow = 2.166}$ a byla vypočítána stejně jako $K_\uparrow$. Průměr těchto dvou hodnot je tedy:

        \begin{equation*}
            \overline{K} = \dfrac{K_\downarrow + K_\uparrow}{2} = \dfrac{2.155 + 2.166}{2} \rightarrow \boxed{K = 2.160}
        \end{equation*}
    \subsection{Závěr} 
    Z měření plyne, že závislost mezi relativní změnou elektrického odporu $\Delta R$ a mechanické poměrné deformace $\epsilon$ je skutečně lineární. Díky tomuto zjištění lze určit součinitel deformační citlivosti $K$, který lze spočítat jako směrnice závislosti pomocí metody nejmenších čtverců.

    Součinitel deformační citlivosti $K$ byl z naměřených hodnot určen jako $K = 2.160$, což je o cca $10\%$ více, než je katalogová hodnota~\cite{navod}.
    
\pagebreak

\section{Úkol 3 - Rezonanční kmitočty malých nosníčků}
    Kapitola vychází ze zdroje~\cite{navod}.
    \subsection{Teoretický rozbor}
    Některá tělesa se mohou chovat jako tlumené oscilátory, to jest, že při jejich vychýlení začnou kmitat kolem rovnovážné polohy kmitočtem, který je závislý na materiálu, rozměrech a dalších parametrech.

    Frekvence těchto kmitů se postupně se zmenšující amplitudou kmitů zvyšuje, protože rychlost tělesa je větší jak jeho vychýlení. Těsně po vychýlení těleso začne kmitat kmitočtem rezonančním. Tento kmitočet je určen materiálovými vlastnotmi a rozměry tělesa a slouží jako počáteční frekvence tlumených kmitů tělesa po vychýlení.

    Jelikož tenzometry měří deformaci, tak je pomocí nich možné měřit i frekvenci kmitů. V této úloze se využívá dalšího generátoru harmonických průběhů a pomocí Lissajousových obrazců lze rezonanční kmitočet malých nosníčků určit.

            \printfigure[Přípravek s nosníky a tenzometry~\cite{navod}]{src/nosnik.png}{0.70}{nosnik}
    

    \subsection{Postup měření}
    \begin{enumerate}
        \item K aparatuře Vishay P-3500 byly připojeny tenzometry na malých nosnících.
        \item K osciloskopu byl na první kanál přiveden analogový výstup z aparatury Vishay P-3500 a na druhý kanál byl přiveden signál z generátoru.
        \item Na osciloskopu byl nastaveno zobrazení X-Y, tedy Lissajousových obrazců a pomocí nich byl nalezen rezonanční kmitočet nosníků.
    \end{enumerate}
    \pagebreak
    \subsection{Naměřené hodnoty}

        \begin{table}[!h]
            \centering
            \caption{Naměřené rezonanční frekvencne malých nosníčků}\label{tab:ukol3-frekvence}
            \vspace{0.3cm}
            \begin{tabular}{|c|c|}
                \hline
                $f$[Hz] & 46.95  \\ 
                \hline
                $f$[Hz] & 58.82  \\ 
                \hline
            \end{tabular}
        \end{table}

    \subsection{Závěr} 
    Z naměřených frekvencí lze pozorovat, že menší nosníček má vyšší rezonanční kmitočet. Z geometrických důvodů toto dává smysl, pokud budeme mít jiný přirozený oscilátor, například matematické kyvaldo, tak kyvadlo s menším závěsem bude kmitat rychleji, protože svým koncem při podobné rychlosti musí pro jeden kmit opsat menší dráhu~\cite{mkyvadlo}.

\pagebreak

% Ukol 4 - 
\section{Úkol 4 - Parametry velkého nosníku}

    \subsection{Teoretický rozbor}
    Z mechanických vlastností tuhých těles plyne pomocí Hookova zákona pro nosník vztah:
        \begin{equation}
            \sigma = E \cdot \epsilon \quad [\text{Pa}]
        \end{equation}
        \begin{conditions}
            $\sigma_n$ & mechanické napětí v příslušném směru [Pa] \\
            $E$ & modul pružnosti v tahu [Pa]. Je to konstanta úměrnosti, která je u běžných  \\
                & kovových materiálů směrově nezávislá \\
            $\epsilon$ & je poměrná deformace v příslušném směru [-] 
        \end{conditions}
        Z tohoto vztahu plyne, že nosník se chová přibližně jako pružina a působí proti působící síle.

        Mechanické namáhání se vždy projeví více směry, tudíž když se těleso v jednom směru natahuje, tak se v zájmu zachování konstantího objemu v druhém směru zúžuje.

        Poměr deformací v různých směrech při namáhání v jednom směru se dá vyjádřit pomocí \textit{Poissonova čísla} $\mu$.
        \begin{equation}
            \mu = \dfrac{\epsilon_t}{\epsilon_n} \quad [-]
        \end{equation}
        \begin{conditions}
            $\mu$ & Poissonovo číslo [-] \\
            $\epsilon_t$ & poměrná deformace ve směru kolmém na působící sílu [-] \\
            $\epsilon_n$ & poměrná deformace ve směru působící síly [-]
        \end{conditions}
        K získání mechanického napětí působícího na velký nosník lze postupovat následujícím způsobem:
        \begin{align}
            \sigma_n &= \dfrac{M}{W} \quad [\text{Pa}]\\
            W &= \dfrac{1}{6} \cdot b \cdot h^2 \quad [\text{m}^3]\\
            M &= F \cdot r \quad [N\cdot m] 
        \end{align}
        \pagebreak
        \linebreak
        \begin{conditions}
            $\sigma_n$ & mechanické napětí [Pa] \\
            $M$ & ohybový moment [N$\cdot$m] \\
            $W$ & modul průřezu nosníku [m$^3$] \\
            $F$ & síla působící na nosník [N]
        \end{conditions}
    \subsection{Postup měření}
        \begin{enumerate}
            \item K aparatuře Vishay P-3500 byl připojen tenzometr na velkém nosníku měřící podélnou složku poměrné deformace. 
            \item Byla zapsána hodnota zobrazená na displeji pro nezatížený nosník.
            \item Na nosník by zavěšen závěs a zapsána zobrazená hodnota na displeji.
            \item K závěsu bylo přidáno závaží 1 kg a zapsána zobrazená hodnota na displeji.
            \item Tímto zůsobem bylo postupováno s přidáváním závaží vždy po 1 kg až do stavu, kdz na nosníku byl zavěšen závěs a 5 kg.
        \end{enumerate}
    \subsection{Naměřené hodnoty}
        \begin{itemize}
            \item $r = 0.4$ m
            \item $b = 0.08$ m
            \item $h = 0.006$ m
        \end{itemize}
        %   m4 = [0 2.78 3.78 4.78 5.78 6.78 7.78];
        %   delta_l4_podel = [170 388 463 553 630 698 781];
        %   delta_l4_pric = [-156 -209 -231 -254 -277 -298 -320];

        \begin{protocoltable}[Naměřené hodnoty z Vishay P-3500 při postupném zatěžování nosníku]{|c|C|C|C|C|C|C|C|}{ukol4-mereni}
            \hline
            $m$[kg] & 0 & $m_z$ & $m_z+1$ & $m_z+2$ &  $m_z+3$ & $m_z+4$ & $m_z+5$  \\ 
            \hline
            $D_{n}$[$\mu$m$\cdot$m$^{-1}$] & 170 & 388 & 463 & 553 & 630 & 698 & 781  \\  
            \hline
            $D_{t}$[$\mu$m$\cdot$m$^{-1}$] & -156 & -209 & -231 & -254 & -277 & -298 & -320  \\  
            \hline
        \end{protocoltable}

        \pagebreak

    \subsection{Zpracované výsledky měření}
        Jelikož se závislost informace z \textit{Vishay P-3500 D} a zatížení nosníku $m$ jeví jako lineární, tak je možné použít metodu nejmenších čtverců a charakteristiku extrapolovat:

        \printfigure[Závislost údaje Vishay P-3500 $D$ na zatížení nosníku $m$]{src/zatizeni.png}{0.35}{charka}


        Naměřené body byly posunuty tak, aby byl v nule stav, kdy byl nosník zatížen pouze prádzným závěsem. Ve chvíli, kdy se vykreslí přímka s konstantní hodnotou $D$, která byla naměřena při nezatíženém nosníku, tak lze jejím průnikem s charakteristikou určit hmotnost závěsu jako $\boxed{m_z = 2.78 \text{ kg}}$.


        Se známými informacemi o zatížení nosníku lze vypočítat mechanické napětí $\sigma_n$:
        \begin{align*}
            M &= F \cdot r = m \cdot g \cdot r = 7.78 \cdot 9.81 \cdot 0.4 = 30.50 \ \text{N} \cdot \text{m} \\
            W &= \dfrac{1}{6} \cdot b \cdot h^2 = \dfrac{1}{6} \cdot 0.08 \cdot 0.006^2 = 4.8 \cdot 10^{-7} \ \text{m}^3 \\
            \sigma_n &= \dfrac{M}{W} = \dfrac{10.9}{4.8 \cdot 10^{-7}} = 6.354\cdot 10^7 \ \text{Pa}
        \end{align*}
        

        Následně je možné spočítat poměrnou mechanickou deformaci $\epsilon$ v podélném a příčném směru:
        \begin{equation*}
            \epsilon_{n} = \dfrac{\Delta D_{n} \cdot 10^{-6}}{K} = \dfrac{|D_{n} - D_{n0}| \cdot 10^{-6}}{K} = \dfrac{|781 - 170| \cdot 10^{-6}}{2.16} = 2.829 \cdot 10^{-4} \\
        \end{equation*}

        \begin{equation*}
              \epsilon_{t} = \dfrac{\Delta D_{t} \cdot 10^{-6}}{K} = \dfrac{|D_{t} - D_{t0}| \cdot 10^{-6}}{K} = \dfrac{|-320 + 156| \cdot 10^{-6}}{2.16} = 7.591 \cdot 10^{-5} 
        \end{equation*}

        Nyní už je dostupných dost informací k vypočítání modulu pružnosti $E$ a \textit{Poissonova čísla} $\mu$:
        \begin{equation*}
            E = \dfrac{\sigma_n}{\epsilon_n} = \dfrac{6.354\cdot 10^7}{2.829 \cdot 10^{-4}} = 2.243 \cdot 10^{11} \  \text{Pa}
        \end{equation*}

        \begin{equation*}
            \mu = \dfrac{\epsilon_t}{\epsilon_n} = \dfrac{7.591 \cdot 10^{-5}}{2.829 \cdot 10^{-4}} = 0.2684 
        \end{equation*}
        % 0 & 2.270 \cdot 10^{7} & 3.087\cdot 10^{7} & 3.904 \cdot 10^{7} & 4.720 \cdot 10^{7} & 5.537\cdot 10^{7} & 6.354 \cdot 10^{7}
        % 0	& 1.009 & 1.356 & 1.773 & 2.129 & 2.444 & 2.828
        % 0 & 24.53 & 34.71 & 45.36 & 56.01 & 65.73 & 75.91
        % X & 225.0 & 227.6 & 220.2	& 221.6 & 226.6 & 224.7
        \begin{protocoltable}[Vypočtené hodnoty $E$, $\mu$ a veličiny pro to potřebné]{|c|C|C|C|C|C|C|C|}{ukol4-hodnoty}
            \hline
            $m$[kg] & 0 & 2.78 & 3.78 & 4.78 & 5.78 & 6.78 & 7.78  \\ 
            \hline
            $D_{n}$[$\mu$m$\cdot$m$^{-1}$] & 170 & 388 & 463 & 553 & 630 & 698 & 781  \\  
            \hline
            $D_{t}$[$\mu$m$\cdot$m$^{-1}$] & -156 & -209 & -231 & -254 & -277 & -298 & -320  \\  
            \hline
            $M$[N$\cdot$m] & 0 & 10.90 & 14.82 & 18.74 & 22.66 & 26.58 & 30.50  \\  
            \hline
            $\sigma_n$[GPa] & 0 & $0.0227$ & $0.03087$ & $0.03904$ & $0.04720$ & $0.05537$ & $0.06354$  \\  
            \hline
            $\epsilon_{n}$[$10^{-6}$] & 0	& 100.9 & 135.6 & 177.3 & 212.9 & 244.4 & 282.8  \\  
            \hline
            $\epsilon_{t}$[$10^{-6}$] & 0 & 24.53 & 34.71 & 45.36 & 56.01 & 65.73 & 75.910  \\  
            \hline
            $E$[GPa] & X & 225.0 & 227.6 & 220.2 & 221.6 & 226.6 & 224.7  \\  
            \hline
            $\mu$[-] & X & 0.2431 & 0.2560 & 0.2559 & 0.2630 & 0.2689 &	0.2684  \\  
            \hline
        \end{protocoltable}

        Jelikož je možné spočítat $E$ i $\mu$ pro více hodnot zatížení tak je možné je pro přehlednost zprůměrovat (parametry jsou~konstanty a~neměly by se měnit):

        \begin{align*}
            \overline{E} &= \dfrac{1}{N} \sum_{i = 0}^{n} E_i = \dfrac{1}{6} \sum_{i = 1}^{6} E = 224.3 \text{ GPa} \\
            \overline{\mu} &= \dfrac{1}{N} \sum_{i = 0}^{n} \mu_i = \dfrac{1}{6} \sum_{i = 1}^{6} \mu_i = 0.2592 
        \end{align*}

    \subsection{Závěr} 
        Při postupném zatěžování nosníku a měřením hodnot z aparatury \textit{Vishay P-3500} bylo možné extrapolací naměřených bodů spočítat hmotnost závěsu, která vyšla $m = 2.78\text{ kg}$.

        Z naměřených hodnot bylo možné spočítat model pružnosti nosníku $E$, který vyšel v průměru $E = 224.4 \text{ GPa}$, což se od běžné hodnoty $220\text{ GPa}$ liší o cca $2\  \%$. 
        
        Vypočítáno bylo též \textit{Poissonovo číslo} $\mu$, které vyšlo v průměru $\mu = 0.2592$. Běžná hodnota $\mu$ bývá 0.3, což se od naměřené hodnoty liší o cca $14\  \%$\cite{navod}.
       
\pagebreak

\section{Úkol 5 - Parametry velkého nosníku, můstková měření}
    Kapitola vychází ze zdroje~\cite{navod}.
    \subsection{Teoretický rozbor}
    Kapitola obsahuje stejnou teorii jako kapitola 2 a 4. Došlo pouze k výměně aparatury \textit{Vishay P-3500} za trojici přípravků s odporovými můstky, ze kterých získáváme místo $D$ informaci o elektrickém napětí $U$.
    \subsection{Postup měření}
        \begin{enumerate}
            \item Napájecí napětí můstku byla nastaveno na 2 V.
            \item Pro dvouvodičové zapojení čtvrtmůstku byly změřeny hodnoty napětí na výstupu můstků při postupném zatěžování nosníku viz. úkol 4.
            \item To samé jsme opakovali pro třívodičové zapojení čtvrtmůstku a pro půlmůstek.
        \end{enumerate}
    \subsection{Naměřené hodnoty}
        \begin{protocoltable}[Naměřená napětí z můstků při zatěžování nosníku]{|c|C|C|C|C|C|C|C|}{ukol5-napeti}
            \hline
            $m$[kg] & 0 & 2.78 & 3.78 & 4.78 & 5.78 & 6.78 & 7.78  \\ 
            \hline
            $U_{1/4} (2v)$[mV] & 0.954 & 0.851 & 0.814 & 0.774 & 0.733 & 0.683 & 0.642  \\  
            \hline
            $U_{1/4} (2v)$ s $R$ [mV] & -39.91 & -39.99 & -40.01 & -40.04 & -40.06 & -40.09 & -40.12  \\  
            \hline
            $U_{1/4} (3v)$[mV] & 1.362 & 1.260 & 1.224 & 1.192 & 1.139 & 1.105 & 1.075  \\  
            \hline
            $U_{1/4} (3v)$ s $R$ [mV] & 1.267 & 1.164 & 1.128 & 1.090 & 1.053 & 1.018 & 0.978 \\  
            \hline
            $U_{1/2}$[mV] & -0.119 & -0.251 & -0.297 & -0.346 & -0.396 & -0.442 & -0.492  \\  
            \hline
            $U_{1/2}$ s $R$ [mV] & -0.059 & -0.192 & -0.236 & -0.284 & -0.331 & -0.376 & -0.424  \\  
            \hline
        \end{protocoltable}
    \subsection{Zpracované výsledky měření}
        Hmotnosti závěsu byly pro měření pomocí různých konfigurací můstků stejnou extrapolací pomocí metody nejmenších čtverců stejně jako v kapitole 4.
        \begin{table}[!h]
            \centering
            \caption{Odečtené hodnoty hmotností závěsu pro různé můstky}\label{tab:ukol5-hmotnosti}
            \vspace{0.3cm}
            \begin{tabular}{|c|c|}
                \hline
                Typ můstku & Odečtená hmotnost $m$ [kg]  \\ 
                \hline
                Dvouvodičové zapojení čtvrtmůstku & 2.35  \\ 
                \hline
                -//- s předřadným R & 2.94  \\ 
                \hline
                Třívodičové zapojení čtvrtmůstku & 2.64  \\ 
                \hline
                -//-  s předřadným R & 2.76  \\ 
                \hline
                Půlmůstek & 2.72  \\ 
                \hline
                Půlmůstek s předřadným R & 2.84  \\ 
                \hline
            \end{tabular}
        \end{table}
        Pro přehlednost lze opět hmotnosti průměrovat:
        \begin{equation*}
            \overline{m} = \dfrac{1}{N} \sum_{i = 0}^{n} m_i = \dfrac{1}{6} \sum_{i = 0}^{6} m_i = 2.708 \text{ kg}
        \end{equation*}
        Pro zjištění $E$ je znovu nutné zjisit informaci o $\epsilon$, kterou lze zjistit z rovnice (\ref{hlavni rovnice}), nicméně za změnu odporu dosadíme změnu napětí pomocí vztahu (\ref{deltaR}).
        \begin{equation*}
            4 \cdot \dfrac{|U_2 - U_{20}|}{U_1} = K \cdot \epsilon \rightarrow \epsilon = 4 \cdot \dfrac{|U_2 - U_{20}|}{K \cdot U_1} = 4 \cdot \dfrac{|(1.075 - 1.362) \cdot 10^{-3}|}{2.16 \cdot 2} = 265.7 \cdot 10^{-6}
        \end{equation*}

        Po zjištění informace o $\epsilon$ je nutné brát v potaz druhy můstků. Pokud pracujeme s čtvrtmůstkem, tak není třeba nic měnit, nicméně pokud měříme s půlmůstkem, kdy se využívá obou složek tenzometrů, tak je nutné vztah upravit na:
        \begin{equation*}
         \epsilon = \dfrac{4}{\mu + 1} \cdot \dfrac{|U_2 - U_{20}|}{K \cdot U_1} = \dfrac{4}{0.2592 + 1} \cdot \dfrac{|(-0.424 + 0.059)\cdot 10^{-3}|}{2.16 \cdot 2} = 268.3 \cdot 10^{-6}
        \end{equation*}
        % 0 & 95.35 & 129.6 & 166.6 & 204.6 & 250.9 & 288.8
        % 0	& 74.06 & 92.57 & 120.3 & 138.9 & 166.6 & 194.4
        % 0 & 94.42 & 127.8 & 157.4 & 206.4 & 237.9 & 265.7
        % 0 & 95.35 & 128.7 & 163.9 & 198.1 & 230.5 & 267.5
        % 0 & 97.04 & 130.9 & 166.9 & 203.6 & 237.5 & 274.2
        % 0 & 97.78 & 130.1 & 165.4 & 200.0 & 233.0 & 268.3
        
        \begin{protocoltable}[Vypočtené hodnoty $\epsilon$ z hodnot z tabulky~\ref{tab:ukol5-napeti}]{|c|C|C|C|C|C|C|C|}{ukol5-deformace}
            \hline
            $\sigma_n$[GPa] & 0 & $0.0227$ & $0.03087$ & $0.03904$ & $0.04720$ & $0.05537$ & $0.06354$  \\   
            \hline
            $\epsilon_{1/4} (2v)$[$10^{-6}$] & 0 & 95.35 & 129.6 & 166.6 & 204.6 & 250.9 & 288.8  \\  
            \hline
            $\epsilon_{1/4} (2v)$ s $R$ [$10^{-6}$] & 0	& 74.06 & 92.57 & 120.3 & 138.9 & 166.6 & 194.4  \\  
            \hline
            $\epsilon_{1/4} (3v)$[$10^{-6}$] & 0 & 94.42 & 127.8 & 157.4 & 206.4 & 237.9 & 265.7  \\  
            \hline
            $\epsilon_{1/4} (3v)$ s $R$ [$10^{-6}$] & 0 & 95.35 & 128.7 & 163.9 & 198.1 & 230.5 & 267.5 \\  
            \hline
            $\epsilon_{1/2}$[$10^{-6}$] &  0 & 97.04 & 130.9 & 166.9 & 203.6 & 237.5 & 274.2  \\  
            \hline
            $\epsilon_{1/2}$ s $R$ [$10^{-6}$] & 0 & 97.78 & 130.1 & 165.4 & 200.0 & 233.0 & 268.3  \\  
            \hline
        \end{protocoltable}

        % X & 238.1 & 238.2 & 234.3 & 230.7 & 220.7 & 220.0
        % X & 306.6 & 333.5 & 324.4 & 340.0 & 332.3 & 326.8
        % X & 240.4 & 241.6 & 248.0 & 228.7 & 232.7 & 239.1
        % X & 238.1 & 239.9 & 238.2 & 238.3 & 240.2 & 237.5
        % X & 234.0 & 236.0 & 233.9 & 231.8 & 233.2 & 231.7
        % X & 232.2 & 237.2 & 236.0 & 236.1 & 237.6 & 236.8
        
        Z $\epsilon$ je poté možné dopočítat hodnoty $E$, kdy se za $\sigma_n$ dosazují hodnoty v~úkolu~4:
        \begin{equation*}
            E_{}= \dfrac{\sigma_n}{\epsilon} = \dfrac{0.06354 2\cdot 10^{9}}{265.7 \cdot 10^{-6}} = 239.1 \text{ GPa}
        \end{equation*}


        \begin{protocoltable}[Vypočtené hodnoty $E$ z hodnot z tabulky~\ref{tab:ukol5-deformace}]{|c|C|C|C|C|C|C|C|}{ukol5-moduly}
            \hline
            $E_{1/4} (2v)$[$GPa$] & X & 238.1 & 238.2 & 234.3 & 230.7 & 220.7 & 220.0  \\  
            \hline
            $E_{1/4} (2v)$ s $R$ [$GPa$] & X & 306.6 & 333.5 & 324.4 & 340.0 & 332.3 & 326.8 \\  
            \hline
            $E_{1/4} (3v)$[$GPa$] & X & 240.4 & 241.6 & 248.0 & 228.7 & 232.7 & 239.1  \\  
            \hline
            $E_{1/4} (3v)$ s $R$ [$GPa$] & X & 238.1 & 239.9 & 238.2 & 238.3 & 240.2 & 237.5 \\  
            \hline
            $E_{1/2}$[$GPa$] & X & 234.0 & 236.0 & 233.9 & 231.8 & 233.2 & 231.7 \\  
            \hline
            $E_{1/2}$ s $R$ [$GPa$] & X & 232.2 & 237.2 & 236.0 & 236.1 & 237.6 & 236.8  \\  
            \hline
        \end{protocoltable}

        \pagebreak

        Pro různé můstky vyšly následující průměrné moduly pružnosti $E$:
        
                % 230.3 327.2 238.4 238.7 233.4 236.0
        \begin{table}[!h]
            \centering
            \caption{Výsledné modely pružnosti $E$ pro různé měřící můstky}\label{tab:ukol5-modely-pruznosti}
            \vspace{0.3cm}
            \begin{tabular}{|c|c|}
                \hline
                Typ můstku & Spočtený model pružnosti $E$ [GPa]  \\ 
                \hline
                Dvouvodičové zapojení čtvrtmůstku & 230.3  \\ 
                \hline
                -//- s předřadným R & 327.2  \\ 
                \hline
                Třívodičové zapojení čtvrtmůstku & 238.4  \\ 
                \hline
                -//-  s předřadným R & 238.7  \\ 
                \hline
                Půlmůstek & 233.4  \\ 
                \hline
                Půlmůstek s předřadným R & 236.0  \\ 
                \hline
            \end{tabular}
        \end{table}

        %

    \subsection{Závěr} 

    Z naměřených a vypočtených hodnot lze usoudit, že dvouvodičové zapojení \linebreak čtvrtmůstku je ze všech nejméně přesné a stabilní. Při připojení předřadných odporů vodičů se hodnota modulu pružnosti $E$ změní z 230.3 GPa na 327.2 GPa. Vypočtená hmotnost závěsu se změní na 2.35 kg na 2.94 kg.

    Třívodičové zapojení čtvrmůstku je stabilnější a přesnější. Při změně odporu přívodních vodičů se změní modul pružnosti $E$ z 238.4 GPa na 238.7 GPa, což je nejmenší změna ze všech můstků. Vypočtená hmotnost závěsu se změní na 2.64 kg na 2.76 kg.

    Zapojení půlmůstek používá obě složky tenzometru a při připojení předřadného odporu $R = 50 \ \Omega$ se vypočtená $E$ změní z 233.4 GPa na 236.0 GPa. Tato hodnota je nejblíže referenční 220 GPa. Hmotnost závěsu se změní po připojení předřadného odporu z 2.72 kg na 2.84 kg.

    Průměrná hodnota hmotnosti závěsu 2.708 kg se příliš neliší od 2.78 kg spočítaná v úkolu 4.

       

\pagebreak

\section{Úkol 6 - Nejistota výstupního napětí}
    Kapitola vychází ze zdrojů~\cite{nejistoty},~\cite{nejistoty_prezentace} a~\cite{zaokrouhlovani}.
    \subsection{Teoretický rozbor}
        Pro měření deformace je možné tenzometr připojit do měřícího můstku, který poté změnou svého odporu vytvoří diferenční napětí, které se poté dá změřit multimetrem.

        Tato metoda je poměrně přesná, nicméně všechny vnitřní komponenty můstku mají své tolerance a důsledkem toho je nejistota výstupního napětí, která se dá určit pomocí dílčích nejistot měření komponent můstku a napájecího napětí pomocí věty o šíření nejistoty.
        
        Výstupní napětí měřícího můstku lze popsat následujícím vztahem:
        \begin{equation}
            \label{mustek}
            U = f(U_z, R_1, R_2, R_3, R_t) = U_z \cdot \left( \dfrac{R_1}{R_1+R_2} - \dfrac{R_t}{R_3+R_t} \right) \quad [\text{mV}]
        \end{equation}
    \subsection{Postup měření}
        \begin{enumerate}
            \item Bylo proměřeno napájecí napětí můstku pomocí multimetru.
            \item V konfiguraci třívodičového zapojení čtvrtmůstku bylo změřeno výstupní napětí můstku pomocí multimetru.
            \item Byly změřeny jednotlivé odpory v můstku pomocí multimetru.
            \item Byl změřen odpor nezatíženého tenzometru pomocí multimetru.
        \end{enumerate}

    \subsection{Naměřené hodnoty}
            \begin{table}[!h]
            \centering
            \caption{Naměřené hodnoty výstupního napětí, napájecího napětí a odporů v můstku}\label{tab:nejistota6}
            \vspace{0.3cm}
            \begin{tabular}{|c|c|}
                \hline
                $U$[mV] & 1.445  \\ 
                \hline
                $U_z$[V] & 1.999  \\ 
                \hline
                $R_t$[$\Omega$] & 119.84  \\ 
                \hline
                $R_1$[$\Omega$] & 120.13  \\ 
                \hline
                $R_2$[$\Omega$] & 120.14  \\ 
                \hline
                $R_3$[$\Omega$] & 120.18  \\ 
                \hline
            \end{tabular}
        \end{table}
    \subsection{Zpracované výsledky měření}

    Nejdříve je nutné spočítat $U_{teor}$, aby bylo možné k nějaké hodnotě nejistotu vztáhnout.
    \begin{align*}
        U_{teor} &= U_z \cdot \left( \dfrac{R_1}{R_1+R_2} - \dfrac{R_t}{R_3+R_t} \right)\\
        U_{teor} &= 1.999 \cdot \left( \dfrac{120.13}{120.13+120.14} - \dfrac{119.84}{120.18+199.84} \right) \rightarrow \boxed{U_{teor} = 1.374 \cdot 10^{-3} \ \text{V}}
    \end{align*}

    Dále je nutné určit dílčí nejistoty pro spočtení celkové nejistoty $u_C(U_{teor})$.
    Pro výpočet nejistoty typu B udává výrobce relativní chybu při měření napětí $\delta_U = (0.0035 + 0.0005)\ \%$ a~relativní chybu při~měření odporu $\delta_R = (0.010 + 0.001)\ \%$~\cite{Agilent}. Vztahy pro~nejistoty typu~B pro~měření napětí a~odporu jsou~následující:  
    \begin{align*}
        u_B(U_z) &= \dfrac{\Delta U_z}{\chi} = \dfrac{\dfrac{\delta_{UH}}{100} \cdot U_z +\dfrac{\delta_{UR}}{100} \cdot U_R}{\chi} \\
        u_B(U_z) &= \dfrac{\dfrac{0.0035}{100} \cdot 1.999 +\dfrac{0.0005}{100} \cdot 10}{\sqrt{3}} = 6.926 \cdot 10^{-5} \ \text{V}
    \end{align*}


    \begin{align*}
        u_B(R_1) &= \dfrac{\Delta R_1}{\chi} = \dfrac{\dfrac{\delta_{RH}}{100} \cdot R +\dfrac{\delta_{RR}}{100} \cdot U_R}{\chi} \\
        u_B(R_1) &= \dfrac{\dfrac{0.010}{100} \cdot 120.13 +\dfrac{0.001}{100} \cdot 1000}{\sqrt{3}} = 1.270 \cdot 10^{-2} \ \text{V}
    \end{align*}

    Pro výpočet celkové nejistoty je nutné jednotlivé nejistoty vynásobit citlivostními koeficienty $A$. Jejich výpočet je parciální derivací vztahu (\ref{mustek}) podel jednotlivých veličin, do kterých se dosadí hodnoty z tabulky \ref{tab:nejistota6}. 
    \begin{align*}
        A_{U_z} &=  \dfrac{\partial f}{\partial U_z} = \left( \dfrac{R_1}{R_1+R_2} - \dfrac{R_t}{R_3+R_t} \right) = 6.875 \cdot 10^{-4}\\
        A_{R_1} &=  \dfrac{\partial f}{\partial R_1} = U_z \cdot \dfrac{R_2}{{(R_1 + R_2)^2}} = 4.160\cdot 10^{-3} \ \text{V} \cdot \Omega^{-1} \\
        A_{R_2} &=  \dfrac{\partial f}{\partial R_2} = -U_z \cdot \dfrac{R_1}{{(R_1 + R_2)^2}} = -4.160\cdot 10^{-3} \ \text{V} \cdot \Omega^{-1} \\
         A_{R_3} &=  \dfrac{\partial f}{\partial R_3} = -U_z \cdot \dfrac{R_t}{{(R_3 + R_t)^2}} = -4.158\cdot 10^{-3} \ \text{V} \cdot \Omega^{-1} \\
    \end{align*}
    \begin{align*}
        A_{R_t} &=  \dfrac{\partial f}{\partial R_4} = U_z \cdot \dfrac{R_3}{{(R_3 + R_t)^2}} = 4.170\cdot 10^{-3} \ \text{V} \cdot \Omega^{-1}\\
    \end{align*}
    Jelikož je měření dílčích komponent zatíženo nízkou náhodnou chybou, tak je~možné zavést podmínku $u_A(i) = 0$ a~tím pádem $u_B(i) = u_C(i)$. Nyní je možné pomocí věty o šíření nejistot spočítat celkovou nejistotu $u_C(U_{teor})$.
    \begin{equation*}
        u_c(U_{teor}) = \sqrt{   \sum_{i = 0}^{n} (A_i \cdot u_c(i))^2    } = 9.774 \cdot 10^{-5} \ \text{V}
    \end{equation*}

    K získání rozšířené nejistoty $U(U_{teor})$ je~nutné nejistotu vynásobit koeficientem rozšíření $k_r$.
    \begin{equation*}
        U(U_{teor}) = k_r \cdot u_c(U_{teor}) = 2 \cdot 9.774 \cdot 10^{-5} = 1.955 \cdot 10^{-4} \ \text{V}
    \end{equation*}

    Konečně je možné uvézt zaokrouhlenou hodnotu a nejistotu~\cite{zaokrouhlovani}.
    \begin{equation*}
        U_{teor} = (1.374 \pm 0.1955) \ \text{mV} \rightarrow \boxed{U_{teor} = (1.4 \pm 0.2) \ \text{mV}}
    \end{equation*}

    \begin{protocoltable}[Uncertainty budget pro určení výstupního napětí můstku]{|C|C|c|C|}{ukol6-budget-mustek}
            \hline
            Veličina & Hodnota & $\Delta_{MAX}$ & $\chi$ \\
            \hline 
            $U_z$[V] & $1.999$ V & $1.200 \cdot 10^{-4} \  V $ & $\sqrt{3}$\\
            \hline
            $R_1$[$\Omega$] & $120.13 \ \Omega$  &  $2.201 \cdot 10^{-2} \  \Omega $ & $\sqrt{3}$\\
            \hline
            $R_2$[$\Omega$] & $120.14 \ \Omega$  &  $2.201 \cdot 10^{-2} \  \Omega  $ & $\sqrt{3}$\\
            \hline
            $R_3$[$\Omega$] & $120.18 \ \Omega$  &  $2.201 \cdot 10^{-2} \  \Omega  $ & $\sqrt{3}$\\
            \hline
            $R_t$[$\Omega$] & $119.84 \ \Omega$  & $1.420 \cdot 10^{-2} \  \Omega  $ & $\sqrt{3}$\\
            \hline
            Veličina & $u_{B}$ & $\frac{\partial \rho}{\partial x_i}$ & $u_c$ \\
            \hline
            $U_z$[V] & $9.774 \cdot 10^{-5}$ V & $6.875 \cdot 10^{-4}$  &  $9.774 \cdot 10^{-5}$ V \\
            \hline
            $R_1$[$\Omega$] & $1.271 \cdot 10^{-2} \ \Omega$ & $4.160 \cdot 10^{-3} \ V \cdot \Omega^{-1} $ & $1.271 \cdot 10^{-2} \ \Omega$ \\
            \hline
            $R_2$[$\Omega$] & $1.271 \cdot 10^{-2} \ \Omega$ & $-4.160 \cdot 10^{-3} \  V \cdot \Omega^{-1} $ & $1.271 \cdot 10^{-2} \ \Omega$ \\
            \hline
            $R_3$[$\Omega$] & $1.271 \cdot 10^{-2} \ \Omega$ & $-4.158 \cdot 10^{-3} \  V \cdot \Omega^{-1} $ & $1.271 \cdot 10^{-2} \ \Omega$ \\
            \hline
            $R_t$[$\Omega$] & $8.195 \cdot 10^{-3} \ \Omega$ & $4.170 \cdot 10^{-3} \  V \cdot \Omega^{-1} $ & $8.195 \cdot 10^{-3} \ \Omega$ \\
            \hline
            \multicolumn{2}{|c|}{$U_{teor} = 1.445\cdot 10^{-3}$ V} & \multicolumn{2}{c|}{$u_C = 9.774 \cdot 10^{-5}$ V} \\
            \hline
        \end{protocoltable}

    \subsection{Závěr} 
    Zaokrouhlená hodnota výstupního napětí $U_{teor}$ s nejistotou vyšla $(1.4 \pm 0.2) \ \text{mV}$. Hodnota výstupního napětí $U = 1.445$ mV naměřená multimetrem se nachází v~intervalu teoretické hodnoty s nejistotou.
    \pagebreak

\section{Závěr}
Z měření plyne, že závislost mezi relativní změnou elektrického odporu $\Delta R$ a mechanické poměrné deformace $\epsilon$ je skutečně lineární. Díky tomuto zjištění lze určit součinitel deformační citlivosti $K$, který lze spočítat jako směrnice závislosti pomocí metody nejmenších čtverců. Součinitel deformační citlivosti $K$ byl z naměřených hodnot určen jako $K = 2.160$, což je o cca $10\%$ více, než je katalogová hodnota~\cite{navod}.

Z naměřených frekvencí lze pozorovat, že menší nosníček má vyšší rezonanční kmitočet. Z geometrických důvodů toto dává smysl, pokud budeme mít jiný přirozený oscilátor například matematické kyvaldo, tak kyvadlo s menším závěsem bude kmitat rychleji, protože svým koncem při podobné rychlosti musí pro jeden kmit opsat menší dráhu~\cite{mkyvadlo}.

Při postupném zatěžování nosníku a měřením hodnot z aparatury \textit{Vishay P-3500} bylo možné extrapolací naměřených bodů spočítat hmotnost závěsu, která vyšla $m = 2.78\text{ kg}$. Z naměřených hodnot bylo možné spočítat model pružnosti nosníku $E$, který vyšel v průměru $E = 224.4 \text{ GPa}$, což se od běžné hodnoty $220\text{ GPa}$ liší o~cca $2\  \%$. Vypočítáno bylo též \textit{Poissonovo číslo} $\mu$, které vyšlo v průměru $\mu = 0.2592$. Běžná hodnota $\mu$ bývá 0.3, což se od naměřené hodnoty liší o cca $14\  \%$\cite{navod}.

Z naměřených a vypočtených hodnot lze usoudit, že dvouvodičové zapojení \linebreak čtvrtmůstku je ze všech nejméně přesné a stabilní. Při připojení předřadných odporů vodičů se hodnota modulu pružnosti $E$ změní z 230.3 GPa na 327.2 GPa. Vypočtená hmotnost závěsu se změní na 2.35 kg na 2.94 kg. 

Třívodičové zapojení čtvrmůstku je stabilnější a přesnější. Při změně odporu přívodních vodičů se změní modul pružnosti $E$ z 238.4 GPa na 238.7 GPa, což je~nejmenší změna ze všech můstků. Vypočtená hmotnost závěsu se změní na 2.64 kg na 2.76 kg. 

Zapojení půlmůstek používá obě složky tenzometru a při připojení předřadného odporu $R = 50 \ \Omega$ se vypočtená $E$ změní z 233.4 GPa na 236.0 GPa. Tato hodnota je nejblíže referenční 220 GPa. Hmotnost závěsu se změní po připojení předřadného odporu z 2.72 kg na 2.84 kg.
    
Zaokrouhlená hodnota výstupního napětí $U_{teor}$ s nejistotou vyšla $(1.4 \pm 0.2) \ \text{mV}$. Hodnota výstupního napětí $U = 1.445$ mV naměřená multimetrem se nachází v~intervalu teoretické hodnoty s nejistotou.

    
\pagebreak

% Vishay P-3500, SN. 106013
% P. pro cejchování tenzometrů, SN. DKP1167
% Tenzometry, SN. DKP479
% UNI-T UTD2025C, SN. 2100000787
% Agilent 34401A, SN. MY44002660
% RC generátor Hung Chang 8204A, SN. D7064
% Instek GPD-3303S, SN. GES836987



% Seznam přístrojů
\section{Seznam použitých přístrojů}
    \begin{protocoltable}[Seznam použitých přístrojů]{|C|C|C|}{pristroje}
        \hline
        Typ & Přístroj & Inventární číslo  \\
        \hline
        Tenzometrická aparatura & Vishay P-3500 & SN. 106013 \\
        \hline
        P. pro cejchování tenzometrů &  - & SN. DKP1167 \\
        \hline
        Tenzometry & - & SN. DKP479 \\
        \hline
        Meteostanice & Testo 622 & SN. 39507568/505 \\
        \hline
        Osciloskop & UNI-T UTD2025C & SN. 2100000787 \\
        \hline
        Multimetr & Agilent 34401A & SN. MY44002660\\
        \hline
        RC generátor & Hung Chang 8204A & SN. D7064\\
        \hline
        Lab. zdroj & Instek GPD-3303S & SN. GES836987\\
        \hline


    \end{protocoltable}%

\pagebreak

% Reference
{\printbibliography}

\end{document} % Konec dokumentu
